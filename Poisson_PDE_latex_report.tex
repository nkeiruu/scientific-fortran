\documentclass{article}
\usepackage[utf8]{inputenc}
\usepackage{float}
\usepackage{amsmath}
\usepackage{graphicx}
\title{noo}
\author{nkeiruubadike }
\date{April 2021}

\begin{document}
\section{Theory}
The purpose of this program (entitled $nkeiru\_ubadike\_hw8.f08$) is to solve the Poisson equation given by

\begin{equation}
  \frac{\partial^2 U}{\partial x^2}+\frac{\partial^2 U}{\partial y^2} =4\pi q
  \label{poiss}
\end{equation}

Where U is the potential and q is the charge.

Equation \ref{poiss} is solved numerically using the finite difference method for a grounded metal box. Thus the boundary conditions are U = 0 on the four edges of the box. This method produces a discretized approximation of equation \ref{poiss} given by
\begin{equation}\label{discr}
  \frac{U_{i+1, j} + U_{i-1, j} + U_{i,j+1}U_{i,j-1} -4U_{i,j}}{h^2} = 4\pi q_{i,j}
\end{equation}

Equation \ref{discr} is solved for $U_{i,j}$ and an initial guess $U_{i,j}^0$ is used to obtain a new guess $U_{i,j}$
\begin{equation}
  U_{i,j}^1 = \frac{U_{i + 1,j}^0 + U_{i - 1,j}^0 + U_{i,j+ 1}^0 + U_{i,j-1}^0 - 4\pi h^2 q_{i,j}}{4}   
\end{equation}
This process is repeated until the solution converges. This method is known as the relaxation method. The convergence criteria is $max(\lvert(U_{i,j}^0 - U_{i,j}^1\rvert)<  accuracy $ 
\section{Input Parameters}

The number of cells in the i and j direction is denoted as $\mathbf{n_i}$ and $\mathbf{n_j}$ respectively. In this case, $n_i = n_j$ = 20 Each cell is a square with $\mathbf{h__{size}}$ = 1.0. The charge density $\mathbf{q__den}$, is a gradient of electric charge that is increasing the in the positive i direction and is constant in the j direction.  The accuracy = 1.0 $\times 10 ^-5$ determines when our solution converges.

\begin{figure}[h]
  \begin{center}
  \includegraphics[width = 0.50\textwidth]{plot_density.png}
 % \caption{Charge Density}
  \label{q_density} Charge density plot
\end{center}
\end{figure}

Our charge density is pictured in figure \ref{q_density}.

\section{Results}

%he results are pictured in figure \ref{potential_plot}

\begin{figure}[H]
  \begin{center}
  \includegraphics[width = 0.5\textwidth]{plot_poisson.png}
  \caption{\label{potential_plot} Potential density plot}
\end{center}
\end{figure}

The greatest region of potential is negative and is centered to the left of the box. The boundary conditions, namely U = 0 on all four edges, are clearly pictured in Figure \ref{potential_plot}.


\end{document}
